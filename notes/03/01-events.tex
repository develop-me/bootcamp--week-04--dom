\section{Event Driven Programming}

Almost everything you do in JavaScript will be a response to an event.
\\

Some examples:

\begin{itemize}
\item the page loading
\item user clicking an element
\item user submitting a form
\item user moving the mouse
\item resizing the window
\end{itemize}

Events allow us to respond to a user's actions.
\\

We use the \texttt{addEventListener} method to tell the browser what we want to happen when the event is \textbf{triggered}.
\\

\pagebreak

We pass \texttt{addEventListener} a function, which gets called by the browser each time the registered event occurs. We call such a function an \textbf{event handler}:

\begin{minted}{javascript}
    // this runs straight away
    console.log("page loaded");

    let container = document.getElementById("container");

    // this runs when the element is clicked
    container.addEventListener("click", () => console.log("clicked"));

    // this runs when the mouse moves over the element
    container.addEventListener("mousemove", () => console.log("mouse moving"));
\end{minted}

There are all sorts of events:

\begin{itemize}
    \item \texttt{keydown}: when a key is pressed down
    \item \texttt{keyup}: when the key is released
    \item \texttt{click}: when the element is clicked
    \item \texttt{mousedown}: when the mouse is pressed down
    \item \texttt{mouseup}: when it comes back up again
    \item \texttt{focus}/\texttt{blur}: when a form field gets/loses focus
    \item \texttt{change}/\texttt{input}: fires when a form input changes
    \item \texttt{submit}: fired on submitting a form
\end{itemize}

A full list is available on the \href{https://developer.mozilla.org/en-US/docs/Web/Events}{MDN site}
\\

Some events will only apply to specific types of elements (e.g. you can only submit a form)

\pagebreak

\section{Window Events}

Some events need to be on the \texttt{window}: most usefully, resizing and scrolling the page.

\begin{minted}{javascript}
    window.addEventListener("scroll", () => {
        console.log("scrolling");
    });

    window.addEventListener("resize", () => {
        console.log("resizing");
    });
\end{minted}



\section{State}

Event handlers are \textbf{short-lived}: they are triggered by an event, run the code inside, and then they're done. Any variables that are declared \textit{inside} an event handler will only exist temporarily.
\\

Your main application code is (comparatively) \textbf{long-lived}: any variables will exist as long as the page isn't refreshed.
\\

This means if we want to keep track of any values \textit{between} events, we need to make sure the variables live \textit{outside} our event handlers. This is what we refer to as \textbf{state}: variables we use to keep track of changes in our app.

\begin{minted}{javascript}
    // long-lived variables
    let increment = document.getElementById("increment");

    // the state
    // keep track of a value that changes over time
    let counter = 0;

    // event handlers need to refer to variables outside their local scope
    // otherwise they can't keep track of anything
    increment.addEventListener("click", () => counter += 1);
\end{minted}

Remember, \textit{you should avoid querying the DOM if you can}. Keep your state in JavaScript and use variables to keep track of changes.



\section{Additional Resources}

\begin{itemize}[leftmargin=*]
    \item \href{https://developer.mozilla.org/en-US/docs/Web/API/EventTarget/addEventListener}{MDN: \texttt{addEventListener}}
    \item \href{https://goo.gl/cn9pWm}{State}
    \item \href{https://eloquentjavascript.net/14_event.html}{Eloquent JavaScript: Events}
    \item \href{https://matthewrayfield.com/goodies/popup-trombone/}{Pop-Up Trombone}: the best use of the \texttt{resize} event ever
\end{itemize}
