\section{Querying Elements}

You can find out all sorts of information about an element:

\begin{minted}{javascript}
    let container = document.getElementById("container");

    // the height of the element (in pixels)
    container.clientHeight;
    // the width of the element (in pixels)
    container.clientWidth;
    // the height of the element, including borders (in pixels)
    container.offsetHeight;
    // the width of the element, including borders (in pixels)
    container.offsetWidth;
    // returns the position of the element relative to the page
    container.getBoundingClientRect();
    // the text inside the element
    container.textContent;
    // the inline style border colour
    container.style.borderColor;
\end{minted}

You can store these values in variables to use later:

\begin{minted}{javascript}
    let height = container.clientHeight;
    // add 300 pixels to the height of the container
    container.style.height = (height + 300) + "px";
\end{minted}

\textit{It's always better to not query the DOM if you can avoid it.} If you can use variables to keep track of changes your code will be much easier to understand and it will run faster.



\section{\texttt{document} and \texttt{window} Properties}

You can find useful information out about the page\footnote{The \href{https://developer.mozilla.org/en-US/docs/Web/API/Window}{\texttt{window} object} is part of the \textbf{BOM} (Browser Object Model). As well as having the window specific properties/methods, it actually represents the \textbf{global scope} in a browser.}:

\begin{minted}{javascript}
    window.pageYOffset; // the current vertical scroll position
    window.pageXOffset; // the current horizontal scroll position

    window.innerHeight; // the height of the viewport
    window.innerWidth; // the width of the viewport

    document.body.offsetHeight; // the height of the document
    document.body.offsetWidth; // the width of the document
\end{minted}





\section{Manipulating Elements}

You can also edit the styling of the element directly by setting sub-properties of the \texttt{style} property\footnote{This is part of the \textbf{CSSOM} (\href{https://css-tricks.com/an-introduction-and-guide-to-the-css-object-model-cssom}{CSS Object Model}) - we do like our object models in JavaScript}:

\begin{minted}{javascript}
    let container = document.getElementById("container");

    container.style.border = "1px solid red";
    container.style.position = "absolute";

    // move element around the page
    container.style.transform = "translateX(20px)";
    container.style.transform = "translateY(20px)";
    container.style.transform = "translate(20px, 20px)"; // (X, Y)

    // notice we have to write marginTop, not margin-top
    // due to property naming rules
    container.style.marginTop = "20px";
\end{minted}

\pagebreak

You can set the height and width too:

\begin{minted}{javascript}
    let container = document.getElementById("container");

    // make sure you include the units (px in this case)
    container.style.height = "200px";
    container.style.width = "200px";
\end{minted}

It is preferable to use CSS classes where you can, but sometimes you will need to use the \texttt{.style} property if the styling is dependent on values calculated by JavaScript.
\\

You can also change the text inside an element:

\begin{minted}{javascript}
    let title = document.getElementById("title");
    title.textContent = "New Title"; // replaces text of #title
\end{minted}

Be careful, \textit{setting \texttt{.textContent} will remove everything inside the element}.



\section{Attributes}

We can query/edit any attribute of an element:

\begin{minted}{html}
    <input id="age" name="age" value="20" />
\end{minted}

\begin{minted}{javascript}
    let input = document.getElementById("age");

    input.getAttribute("name"); // returns "age"
    input.getAttribute("value"); // returns "20" (as a string)
    input.setAttribute("value", "600"); // set the value attribute to "600"
    input.removeAttribute("disabled"); // remove the disabled attribute
\end{minted}


\pagebreak


\section{Form Fields}

To get the \textit{actual} value of the input (as opposed to the default value), we can use the \texttt{.value} property:

\begin{minted}{javascript}
    input.value; // a string containing the current value of the input
\end{minted}

The \texttt{.value} property will \textit{always} return a string - so be careful if you're doing any addition!
\\

We can also call the \texttt{.focus()} method on a form field to give it focus:

\begin{minted}{javascript}
    input.focus(); // gives the input focus
\end{minted}



\section{Data Attributes}

Sometimes we want to let JavaScript know something about an element that isn't a standard property.  We can use \texttt{data-*} attributes for this.
\\

For example, if we wanted to store additional information about some books:

\begin{minted}{html}
    <ul id="books">
        <li data-id="12" data-author="Marijn Haverbeke">
            Eloquent JavaScript
        </li>
        <li data-id="35" data-author="Douglas Crockford">
            JavaScript: The Good Parts
        </li>
        <li data-id="59" data-author="David Flanagan">
            JavaScript: The Definitive Guide
        </li>
    </ul>
\end{minted}

\pagebreak

We could then access this using the \texttt{dataset} property:

\begin{minted}{javascript}
    let first = document.getElementById("books").firstElementChild;

    if (first) {
        console.log(first.dataset.id);
        console.log(first.dataset.author);
    }
\end{minted}

You should only use \texttt{data-*} attributes as a last resort, as reading from the DOM is slow. Ideally the data would be used and stored only in JavaScript.



\section{Additional Resources}

\begin{itemize}[leftmargin=*]
    \item \href{https://developer.mozilla.org/en-US/docs/Web/API/Window/getComputedStyle}{MDN: \texttt{window.getComputedStyle()}}
    \item \href{https://developer.mozilla.org/en-US/docs/Web/API/Element/getBoundingClientRect}{MDN: \texttt{getBoundingClientRect()}}
    \item \href{https://developer.mozilla.org/en-US/docs/Learn/HTML/Howto/Use_data_attributes}{MDN: \texttt{data-*} Attributes}
    \item \href{https://developer.mozilla.org/en-US/docs/Web/API/Window}{MDN: \texttt{window}}
    \item \href{https://developer.mozilla.org/en-US/docs/Web/API/CSS_Object_Model/Determining_the_dimensions_of_elements}{MDN: Determining the dimensions of elements}
\end{itemize}
